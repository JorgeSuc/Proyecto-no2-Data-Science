\documentclass[]{article}
\usepackage{lmodern}
\usepackage{amssymb,amsmath}
\usepackage{ifxetex,ifluatex}
\usepackage{fixltx2e} % provides \textsubscript
\ifnum 0\ifxetex 1\fi\ifluatex 1\fi=0 % if pdftex
  \usepackage[T1]{fontenc}
  \usepackage[utf8]{inputenc}
\else % if luatex or xelatex
  \ifxetex
    \usepackage{mathspec}
  \else
    \usepackage{fontspec}
  \fi
  \defaultfontfeatures{Ligatures=TeX,Scale=MatchLowercase}
\fi
% use upquote if available, for straight quotes in verbatim environments
\IfFileExists{upquote.sty}{\usepackage{upquote}}{}
% use microtype if available
\IfFileExists{microtype.sty}{%
\usepackage{microtype}
\UseMicrotypeSet[protrusion]{basicmath} % disable protrusion for tt fonts
}{}
\usepackage[margin=1in]{geometry}
\usepackage{hyperref}
\hypersetup{unicode=true,
            pdftitle={Proyecto 2 - Analisis Exploratorio},
            pdfauthor={Manuel Valenzuela 15072, Davis Alvarez 15842, Jorge Súchite 15293},
            pdfborder={0 0 0},
            breaklinks=true}
\urlstyle{same}  % don't use monospace font for urls
\usepackage{color}
\usepackage{fancyvrb}
\newcommand{\VerbBar}{|}
\newcommand{\VERB}{\Verb[commandchars=\\\{\}]}
\DefineVerbatimEnvironment{Highlighting}{Verbatim}{commandchars=\\\{\}}
% Add ',fontsize=\small' for more characters per line
\usepackage{framed}
\definecolor{shadecolor}{RGB}{248,248,248}
\newenvironment{Shaded}{\begin{snugshade}}{\end{snugshade}}
\newcommand{\KeywordTok}[1]{\textcolor[rgb]{0.13,0.29,0.53}{\textbf{#1}}}
\newcommand{\DataTypeTok}[1]{\textcolor[rgb]{0.13,0.29,0.53}{#1}}
\newcommand{\DecValTok}[1]{\textcolor[rgb]{0.00,0.00,0.81}{#1}}
\newcommand{\BaseNTok}[1]{\textcolor[rgb]{0.00,0.00,0.81}{#1}}
\newcommand{\FloatTok}[1]{\textcolor[rgb]{0.00,0.00,0.81}{#1}}
\newcommand{\ConstantTok}[1]{\textcolor[rgb]{0.00,0.00,0.00}{#1}}
\newcommand{\CharTok}[1]{\textcolor[rgb]{0.31,0.60,0.02}{#1}}
\newcommand{\SpecialCharTok}[1]{\textcolor[rgb]{0.00,0.00,0.00}{#1}}
\newcommand{\StringTok}[1]{\textcolor[rgb]{0.31,0.60,0.02}{#1}}
\newcommand{\VerbatimStringTok}[1]{\textcolor[rgb]{0.31,0.60,0.02}{#1}}
\newcommand{\SpecialStringTok}[1]{\textcolor[rgb]{0.31,0.60,0.02}{#1}}
\newcommand{\ImportTok}[1]{#1}
\newcommand{\CommentTok}[1]{\textcolor[rgb]{0.56,0.35,0.01}{\textit{#1}}}
\newcommand{\DocumentationTok}[1]{\textcolor[rgb]{0.56,0.35,0.01}{\textbf{\textit{#1}}}}
\newcommand{\AnnotationTok}[1]{\textcolor[rgb]{0.56,0.35,0.01}{\textbf{\textit{#1}}}}
\newcommand{\CommentVarTok}[1]{\textcolor[rgb]{0.56,0.35,0.01}{\textbf{\textit{#1}}}}
\newcommand{\OtherTok}[1]{\textcolor[rgb]{0.56,0.35,0.01}{#1}}
\newcommand{\FunctionTok}[1]{\textcolor[rgb]{0.00,0.00,0.00}{#1}}
\newcommand{\VariableTok}[1]{\textcolor[rgb]{0.00,0.00,0.00}{#1}}
\newcommand{\ControlFlowTok}[1]{\textcolor[rgb]{0.13,0.29,0.53}{\textbf{#1}}}
\newcommand{\OperatorTok}[1]{\textcolor[rgb]{0.81,0.36,0.00}{\textbf{#1}}}
\newcommand{\BuiltInTok}[1]{#1}
\newcommand{\ExtensionTok}[1]{#1}
\newcommand{\PreprocessorTok}[1]{\textcolor[rgb]{0.56,0.35,0.01}{\textit{#1}}}
\newcommand{\AttributeTok}[1]{\textcolor[rgb]{0.77,0.63,0.00}{#1}}
\newcommand{\RegionMarkerTok}[1]{#1}
\newcommand{\InformationTok}[1]{\textcolor[rgb]{0.56,0.35,0.01}{\textbf{\textit{#1}}}}
\newcommand{\WarningTok}[1]{\textcolor[rgb]{0.56,0.35,0.01}{\textbf{\textit{#1}}}}
\newcommand{\AlertTok}[1]{\textcolor[rgb]{0.94,0.16,0.16}{#1}}
\newcommand{\ErrorTok}[1]{\textcolor[rgb]{0.64,0.00,0.00}{\textbf{#1}}}
\newcommand{\NormalTok}[1]{#1}
\usepackage{graphicx,grffile}
\makeatletter
\def\maxwidth{\ifdim\Gin@nat@width>\linewidth\linewidth\else\Gin@nat@width\fi}
\def\maxheight{\ifdim\Gin@nat@height>\textheight\textheight\else\Gin@nat@height\fi}
\makeatother
% Scale images if necessary, so that they will not overflow the page
% margins by default, and it is still possible to overwrite the defaults
% using explicit options in \includegraphics[width, height, ...]{}
\setkeys{Gin}{width=\maxwidth,height=\maxheight,keepaspectratio}
\IfFileExists{parskip.sty}{%
\usepackage{parskip}
}{% else
\setlength{\parindent}{0pt}
\setlength{\parskip}{6pt plus 2pt minus 1pt}
}
\setlength{\emergencystretch}{3em}  % prevent overfull lines
\providecommand{\tightlist}{%
  \setlength{\itemsep}{0pt}\setlength{\parskip}{0pt}}
\setcounter{secnumdepth}{0}
% Redefines (sub)paragraphs to behave more like sections
\ifx\paragraph\undefined\else
\let\oldparagraph\paragraph
\renewcommand{\paragraph}[1]{\oldparagraph{#1}\mbox{}}
\fi
\ifx\subparagraph\undefined\else
\let\oldsubparagraph\subparagraph
\renewcommand{\subparagraph}[1]{\oldsubparagraph{#1}\mbox{}}
\fi

%%% Use protect on footnotes to avoid problems with footnotes in titles
\let\rmarkdownfootnote\footnote%
\def\footnote{\protect\rmarkdownfootnote}

%%% Change title format to be more compact
\usepackage{titling}

% Create subtitle command for use in maketitle
\newcommand{\subtitle}[1]{
  \posttitle{
    \begin{center}\large#1\end{center}
    }
}

\setlength{\droptitle}{-2em}

  \title{Proyecto 2 - Analisis Exploratorio}
    \pretitle{\vspace{\droptitle}\centering\huge}
  \posttitle{\par}
    \author{Manuel Valenzuela 15072, Davis Alvarez 15842, Jorge Súchite 15293}
    \preauthor{\centering\large\emph}
  \postauthor{\par}
      \predate{\centering\large\emph}
  \postdate{\par}
    \date{23 de agosto de 2018}

\usepackage{tabularx}

\begin{document}
\maketitle

\section{Descripción del Tema}\label{descripcion-del-tema}

Para este segundo proyecto propone observar la población de los paises
del mundo a lo largo de los años y así mismo su área superficial, con
estos datos obtener una densidad poblacional y ver si esta esta
relacionada de alguna manera con la felicidad de las personas. Con estos
datos se tratará de hacer 2 pronósticos. El primero de la población en
base a su propio comportamiento; y el segundo, de la felicidad del mundo
en base a su propio comportamiento y así mismo a la densidad poblacional
a lo largo del tiempo.

Para hacer esto se uilizarán 3 datasets, el primero es
\href{https://www.kaggle.com/unsdsn/world-happiness/home}{World Happiness Report},
el cual es un dataset que contiene una puntuación de felicidad anotada
según producción económica, apoyo social, etc. El segundo es
\href{https://www.kaggle.com/centurion1986/countries-population}{Countries Population},
el cual contiene la población de cada país a lo largo de los años,y por
último, el tercero sería
\href{https://www.kaggle.com/fernandol/countries-of-the-world}{Countries of the World},
del cual se obtendría el área superficial de cada país.

\section{Problema Científico}\label{problema-cientifico}

Lo que se buscará realizar con este proyecto es observar la densidad
poblacional y la felicidad de las personas de cada país a lo largo del
tiempo, ver si tienen algún tipo de relación entre sí y tratar de
pronosticar que tan felices serán las personas en los años próximos. Así
mismo, observando el comportamiento de la población a lo largo de los
años se buscará pronosticar la población de los paises en los próximos
años.

\section{Objetivos}\label{objetivos}

\begin{itemize}
     \item Encontrar la Densidad Poblacional de los paises del mundo a lo largo de los años.
     \item Encontrar si hay una relación entre la densidad poblacional y la felicidad de las personas.
     \item Pronosticar la felicidad de las personas por país en los próximos años.
     \item Pronosticar la población por país en los próximos años.
\end{itemize}

\section{Descripción de los Datos}\label{descripcion-de-los-datos}

A continuación se describen los datasets y así mismo cada una de las
variables que estos tienen. Luego de eso se explicará el proceso de
limpieza que se llevó a cabo para tomar en cuenta solo las variables que
serán de utilidad para cumplir los objetivos planteados.

\subsection{World Happiness Report}\label{world-happiness-report}

El World Happiness Report es una encuesta histórica sobre el estado de
la felicidad global. El primer informe se publicó en 2012, el segundo en
2013, el tercero en 2015 y el cuarto en la Actualización de 2016. The
World Happiness 2017, que clasifica a 155 países por su nivel de
felicidad, fue lanzado en las Naciones Unidas en un evento que celebra
el Día Internacional de la Felicidad el 20 de marzo.

Los rankings y puntajes de felicidad utilizan data de una encuesta de
Gallup World. Las puntuaciones son basadas en las respuestas a las
preguntas de evaluación de vida contenidas en esta encuesta. Estas
preguntas se conocen como la Escalera de Cantril, esta les pide a los
encuestados que piensen en una escalera con la mejor vida posible para
ellos siendo un 10 y la peor vida posible siendo un 0. A continuación se
describen las variables contenidas en este dataset.

\begin{center}
\begin{tabularx}{\linewidth}{ |c|X|} 
 \hline
 \textbf{No.} & \textbf{Variable} & \textbf{Tipo}& \textbf{Descripción}\\
 \hline
  1 & Country & Nombre del país\\
 \hline
  2 & Region & Region a la que pertenece\\
 \hline
  3 & Happiness.Rank & Rango del país basado en el Happiness Score \\
 \hline
  4 & Happiness.Score & Puntuación obtenida de Encuesta \\
 \hline
  5 & Standard.Error & La desviación Estándar del Happiness Score \\
 \hline
  6 & Economy..GDP.per.Capita & El grado en que el Producto Interno Bruto del país contribuye al cálculo del puntaje de felicidad. \\
 \hline
  7 & Family & El grado en que la familia contribuye al cálculo del puntaje de felicidad. \\
 \hline
  8 & Health..Life.Expectancy & el grado en que la esperanza de vida contribuye al cálculo del puntaje de felicidad \\
 \hline
 9 & Freedom & La medida en que la libertad contribuye al cálculo de la puntuación de la felicidad \\
 \hline
 10 & Trust..Government.Corruption. & El grado en que la percepción de la corrupción contribuye al puntaje de felicidad \\
 \hline
 11 & Generosity & El grado en que generosidad contribuye al cálculo del puntaje de felicidad \\
 \hline
 12 & Dystopia.Residual & La medida en que la distopía residual contribuyó al cálculo de la puntuación de la felicidad \\
 \hline
 13 & Year & Año en el que fue hecha la encuesta \\
 \hline
 14 & Lower.Confidence.Interval & Intervalo de confianza más bajo del puntaje de felicidad \\
 \hline
 15 & Upper.Confidence.Interval & Intervalo de confianza superior del puntaje de felicidad \\
 \hline
 16 & Whisker.high & Bigote Superior \\
 \hline
 17 & Whisker.low & Bigote Inferior \\
 
 \hline
 \hline
\end{tabularx}
\end{center}

De las variables descritas en la tabla anterior se tenían 3 datasets,
uno por cada año. El proceso de limpieza que se hizo con estos datos fue
cuadrar las variables en todos los datasets, es decir, que se hizo que
todos tuviesen las mismas variables con el mismo nombre. Luego se
unieron los 3 datasets dejando al año como una variable más.

\subsection{Countries Population}\label{countries-population}

Este Dataset contiene la población de 217 paises a lo largo de 56 años,
desde 1960 hasta el 2016. A continuación se detallan las variables de
este dataset.

\begin{center}
\begin{tabularx}{\linewidth}{ |c|X|} 
 \hline
 \textbf{Variable} & \textbf{Descripción}\\
 \hline
 i..Country & ID del país\\
 \hline
 Country.Code & Código del país\\
 \hline
 Indicator.Name & Nombre del indicador \\
 \hline
 Indicator.Code & Código del indicador \\
 \hline
 year & Año de medida \\
 \hline
 population & Población \\
 \hline
 \hline
\end{tabularx}
\end{center}

El proceso de limpieza que se hizo con este dataset fue poner los años
como valor en una fila ya que estos venían como columnas.

\subsection{Countries of the World}\label{countries-of-the-world}

Este dataset contiene información sobre población, región, tamaño del
área, mortalidad infantil y más. Todos estos conjuntos de datos están
formados por datos del gobierno de EE. UU. Las variables contenidas se
detallan a continuación.

\begin{center}
\begin{tabularx}{\linewidth}{ |c|X|} 
 \hline
 \textbf{Variable} & \textbf{Descripción}\\
 \hline
 Country & Nombre del país\\
 \hline
 Region & Región en la que se encuentra el país\\
 \hline
 Population & Población Actual \\
 \hline
 Area..sq..mi.. & Área superficial en $mi^2$ \\
 \hline
 Pop..Density..per.sq..mi.. & Densidad Poblacional en $personas/mi^2$ \\
 \hline
 Coastline..coast.area.ratio. & línea costera \\
 \hline
  Net.migration & Migración neta \\
 \hline
 Infant.mortality..per.1000.births. & Mortalidad infantil (por 1000 nacimientos)\\
 \hline
 GDP....per.capita. & Producto Interno Bruto \\
 \hline
 Literacy.... & Porcentaje de Alfabetismo \\
 \hline
 Phones..per.1000. & Teléfonos por cada 1000 personas \\
 \hline
 Arable.... & Porcentaje de tierra cultivable \\
 \hline
 Crops.... & Porcentaje de cultivos \\
 \hline
 Other.... & Otros \\
 \hline
 Climate & Clima\\
 \hline
 Birthrate & Tasa de Nacimiento\\
 \hline
 Deathrate & Tasa de Mortalidad \\
 \hline
 Agriculture & Agricultura \\
 \hline
 Industry & Industria \\
 \hline
 Service & Servicio \\
 \hline
 \hline
\end{tabularx}
\end{center}

Para este dataset no fue necesario llevar a cabo ningún proceso de
limpieza.

\section{Exploración de Datos}\label{exploracion-de-datos}

\begin{Shaded}
\begin{Highlighting}[]
\KeywordTok{View}\NormalTok{(hWorld)}
\end{Highlighting}
\end{Shaded}

\section{Conclusiones y Hallazgos}\label{conclusiones-y-hallazgos}


\end{document}
